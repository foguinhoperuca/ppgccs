%% Copyright (C)  2013 Jefferson Campos.
%% Permission is granted to copy, distribute and/or modify this document
%% under the terms of the GNU Free Documentation License, Version 1.3
%% or any later version published by the Free Software Foundation;
%% with no Invariant Sections, no Front-Cover Texts, and no Back-Cover Texts.
%% A copy of the license is included in the section entitled "GNU
%% Free Documentation License".

\documentclass[a4paper,12pt]{article}
\usepackage[brazilian]{babel}
\usepackage[utf8]{inputenc}
\usepackage[T1]{fontenc}
\usepackage{hyperref}
\usepackage{url}
\usepackage{amssymb}
\usepackage{amsthm}
\usepackage[pdftex]{graphicx}
\newcommand{\HRule}{\rule{\linewidth}{0.5mm}}

\newtheorem{lazy}{Proposição}

\begin{document}

% Nesta seção serão capturadas as principais idéias do texto
\section{Parte 2 da Aula.}

% TODO colocar na forma do latex.

% # Problema de Fluxo Máximo.

% Pesquisar e detalhar...

% # Problemas em Grafo.

% Problemas de fluxo máximo e corte mínimo. Os problemas abaixo são, em essência, o mesmo.

% ## Logística de Distribuição.

% Resolver com o "problema" de fluxo máximo.

% Institivamente aceitamos que o corte com a menor capacidade é o gargalo que leva o produto da fonte para o sorvedouro.

% ## Rede de Comunicação.

% Exemplo: em um cenário de guerra, derrubar um ou mais links (minimo possível) do inimigo.

% Com grafos é possível utilizar o problema de corte mínimo.

% ## Computação Distribuída em 2 Processadores.

% Dividir o processamento em 2 processadores a fim de minimizar o custo de computação. Considerar o cuso de computação e o custo de comunição entre os processos.

% ### Modelagem Proposta.

% * P_1 := vértice especial representando o processador 1;
% * P_2 := vértice especial representando o processador 2;
% * Demais Vérteice representando os processos;
% * As arestas representam os custos de processamento com o respectivo processador;
% * As arestas entre os processos represnetam o custo de comunicação;

% ## Exercício.

% Uma variação natural: quando há diversas fontes e sorvedouros, consigo utilizar a mesma modelagem?
% De certa forma sim: todo mundo que for fonte na rede eu ligo com um novo vértice que será a fonte do novo grafo. A mesma idéia com o sorvedouro: todo mundo que for sorvedouro eu coloco ligado à um novo vértice que será o sorvedouro do novo grafo. Ligo a nova fonte/sorvedouro com valor infinito aos respectivas fontes/sorvedouros.

% ### Como eu mapearia o valor infinito?

% Minha intuição diz que o valor inicial é 0 para não influenciar na conta do corte mínimo. -1 iria, certamente, influenciar no cálculo.

\section{Definições.}

$G = (N, A)$\\
$N$ é o conjunto de nós;\\
$A$ é o conjunto de arcos;

$u_{i j} >= 0 \forall (i, j) \in A$.\\

$s$ é a fonte\\
$t$ é o sorvedouro\\

$max(v) sujeito a \sum{x_{ij}} - \sum{x_{ji}}$

$j: \{(i, j) \in A \}$
$j: \{(j, i) \in A \}$

isso tudo temos:

$v se i = s;$
$0 se i \in N - \{ s, t \}$
$-v se i = t$

$0 \leq x_{ij} \leq u_{ij} \forall (i, j) \in A.$

\subsection{Rede Residual.}

$r_{ij} = u_{ij} - x_{ij} + x_{ji}$

Caminhos aumentantes é utilizado para identificar o caminho mínimo.

A importância do caminho residual é dar a capacidade de avaliar e mudar o caminho escolhido à produca da otimização do problema.

Algoritmo de busca em largura é mais eficiente no cálculo de caminhos aumentantes. Verificar...

\section{Exercícios Para Próxima Aula.}

É necessário o capítulo do livro (qual) para identificar a notação utilizada.

Exercícios:

\begin{itemize}
\item 6.1
\item 6.3
\item 6.25
\end{itemize}

Exercícios que serão cobrados posteriormente:

\begin{itemize}
\item 6.12
\item 6.15
\end{itemize}

\section{Comentários Jefferson}



% \nocite{*}
% \bibliographystyle{ieeetr}
% \bibliography{efficiency_issues_in_the_RLF_heuristic_for_graph_coloring}

\end{document}
