%% Copyright (C)  2012 Jefferson Campos
%% Permission is granted to copy, distribute and/or modify this document
%% under the terms of the GNU Free Documentation License, Version 1.3
%% or any later version published by the Free Software Foundation;
%% with no Invariant Sections, no Front-Cover Texts, and no Back-Cover Texts.
%% A copy of the license is included in the section entitled "GNU
%% Free Documentation License".

\documentclass[a4paper,12pt]{article}
\usepackage[brazilian]{babel}
\usepackage[utf8]{inputenc}
\usepackage[T1]{fontenc}
\usepackage{hyperref}
\usepackage{url}
\usepackage{amsmath}
\usepackage{amssymb}
\usepackage[pdftex]{graphicx}
\newcommand{\HRule}{\rule{\linewidth}{0.5mm}}

\title{Aula 03.}
\author{Jefferson Campos}
\date{October 2012}

\begin{document}

% \maketitle

\section{Breve Revisão de Sistemas Lineares e Operadores Lineares.}

A aplicação de um operador O em uma função S(t) será denotado por:

$$ O[S(t)] $$

Mais ainda, dizemos que o operador linear uma vez que:

$ O[\alpha \cdot x(t) + \beta y(t)] = \alpha \cdot O[x(t)] + \beta \cdot O[y(t)] $ $ \forall \alpha,\beta$ e funções contínuas $x = x(t), y= y(t) $

Particularmente, uma classe de operadores lineares muito importante é definida como:

$$ z(t) = \int_{-\infty}^{+\infty }{x(t')h(t_1t') dt'} $$

onde h é chamada de resposta impulsional.

Na verdade, se o sistema for linear, ele pode ser completamente caracterizado pela sua resposta a função impulso. A função impulso é dada por

\begin{equation*}
|\delta|= 
\begin{cases}
1, & \text{ se } t= 0 \\ 
0, & \text{ caso contrário }
\end{cases}
\end{equation*}

Vamos chamar $x(t) = \delta(t - t_0) $ e escrever:

$$ z(t) = \int_{-\infty}^{+\infty }{\delta(t' - t_0) \cdot h(t_1t') dt'} = h(t_1t_0) $$

Para operações invariantes ao deslocamentom o operador linear fica da seguinte maneira:

$$ z(t) = \int_{-\infty}^{+\infty }{x(t') h(t - t') dt'} $$ (integral de convolução)

Denotamos a convolução de x(t) e h(t) por $ z(t) = x(t) \ast h(t) $

O processo de convolução também pode ser definido para sequências discretas: Se denotamos $ x_i = x[i] \cdot \Delta t $ e $ y_i = y[i \cdot \Delta t] $, a convolução entre $ x_i $ e $ h_i $ pode ser escrita como:

$$ y_i = \Delta t \cdot \sum_{k=-\infty}^{+\infty}{x_k \cdot h_i \cdot k} $$ (aproximação discreta da integral de convolução)

Assuma que temos uma função contínua, $ s = s(t) $, definida para um intervalo $ T_1 \leq  t \leq T_2 $. Mais ainda, assuma, que s(t) seja \textit{periódica}. A função s(t) pode ser expressa na seguinte forma:

$$ s(t) = \sum_{k=-\infty}^{+\infty}{z_k \cdot e^{jkwt}}, j = \sqrt{-1}, w_0 = 2 \pi f_0 , T = T_2 - T_1 $$ e os $ z_k $ são coeficientes discretos chamados de \textit{série de Fourier}.

Casa $ z_k $ pode ser obtido por:

$$ z_k = \frac{1}{T} \int_{T_1}^{T_2}{s(t)e^{-jkw_0T}} $$

De fato, $e^{jkw_0t} = \cos{(kw_0t)} + j\sin{(kw_0t)} $

Quando consideramos um sinal contínuo $ S = s(t) $, definido por $ t \epsilon \left( -\infty, +\infty \right) $, isto é, não necessariamente periódico, s(t) possui uma outra representação dada por:

$$ s(t) = \frac{1}{2 \pi} \int_{-\infty}^{+\infty}{S(\omega)e^{jwt} d\omega} $$, onde $ S(\omega) $ é chamada \textit{Transformada de Fourier Contínua}.

A Transformada de Fourier Contínua, $ S(\omega) $, onde $ \omega \epsilon \mathbb{C} $, é dada por:

$$ S(\omega) = \int_{-\infty}^{+\infty}{s(t)e^{+jwt} dt} $$


\end{document}
