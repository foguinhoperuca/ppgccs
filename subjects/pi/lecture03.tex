%% Copyright (C)  2012 Jefferson Campos
%% Permission is granted to copy, distribute and/or modify this document
%% under the terms of the GNU Free Documentation License, Version 1.3
%% or any later version published by the Free Software Foundation;
%% with no Invariant Sections, no Front-Cover Texts, and no Back-Cover Texts.
%% A copy of the license is included in the section entitled "GNU
%% Free Documentation License".

\documentclass[a4paper,12pt]{article}
\usepackage[brazilian]{babel}
\usepackage[utf8]{inputenc}
\usepackage[T1]{fontenc}
\usepackage{hyperref}
\usepackage{url}
\usepackage{amsmath}
\usepackage[pdftex]{graphicx}
\newcommand{\HRule}{\rule{\linewidth}{0.5mm}}


\title{Aula 03.}
\author{Jefferson Campos}
\date{October 2012}

\begin{document}

% \maketitle

\section{Breve Revisão de Sistemas Lineares e Operadores Lineares.}

A aplicação de um operador O em uma função S(t) será denotado por:

$$ O[S(t)] $$

Mais ainda, dizemos que o operador linear uma vez que:

$ O[\alpha \cdot x(t) + \beta y(t)] = \alpha \cdot O[x(t)] + \beta \cdot O[y(t)] $ $ \forall \alpha,\beta$ e funções contínuas $x = x(t), y= y(t) $

Particularmente, uma classe de operadores lineares muito importante é definida como:

$$ z(t) = \int_{-\infty}^{+\infty }{x(t')h(t_1t') dt'} $$

onde h é chamada de resposta impulsional.

Na verdade, se o sistema for linear, ele pode ser completamente caracterizado pela sua resposta a função impulso. A função impulso é dada por

\begin{equation*}
|\delta|= 
\begin{cases}
1, & \text{ se } t= 0 \\ 
0, & \text{ caso contrário }
\end{cases}
\end{equation*}

Vamos chamar $x(t) = \delta(t - t_0) $ e escrever:

$$ z(t) = \int_{-\infty}^{+\infty }{\delta(t' - t_0) \cdot h(t_1t') dt'} = h(t_1t_0) $$

Para operações invariantes ao deslocamentom o operador linear fica da seguinte maneira:

$$ z(t) = \int_{-\infty}^{+\infty }{x(t') h(t - t') dt'} $$

\end{document}
