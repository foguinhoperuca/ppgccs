%% Copyright (C)  2013 Jefferson Campos.
%% Permission is granted to copy, distribute and/or modify this document
%% under the terms of the GNU Free Documentation License, Version 1.3
%% or any later version published by the Free Software Foundation;
%% with no Invariant Sections, no Front-Cover Texts, and no Back-Cover Texts.
%% A copy of the license is included in the section entitled "GNU
%% Free Documentation License".

\documentclass[a4paper,12pt]{article}
\usepackage[brazilian]{babel}
\usepackage[utf8]{inputenc}
\usepackage[T1]{fontenc}
\usepackage{hyperref}
\usepackage{url}
\usepackage{amssymb}
\usepackage{amsthm}
\usepackage[pdftex]{graphicx}
\newcommand{\HRule}{\rule{\linewidth}{0.5mm}}

\newtheorem{lazy}{Proposição}

\begin{document}

% Nesta seção serão capturadas as principais idéias do texto
\section{Resumo.}

GCP = Graph Coloring Problem

O problema de coloração de vértices pode ter, ao menos, duas versões: otimização e decisão.

Existem basicamente 3 heurísitcas utilizadas para o GCP, que apesar de terem um desempenho razoável não garantem uma aproximação da solução/qualidade da solução:

\begin{itemize}
\item \textbf{simplest}: ordena os vértices do grafo pelo seu grau e sequencialmente colore cada vértice, de acordo com a ordem pré-estabelecida, utilizando a menor cor possível.
\item \textbf{DSATUR}: Possui o mesmo \lq\lq{}espírito\rq\rq{} do algoritmo \textbf{\lq\lq{}simplest\rq\rq{}} porém com o diferencial na forma de ordenação dos vértices. Esta ordenação utiliza-se do \lq\lq{}grau de saturação\rq\rq{}, isto é, que tem o maior grau induzido pelos vértices já coloridos. % Minha percepção: o fato de induzir pelos vértices já coloridos é justamente retirar os vértices já coloridos porém as demais arestas devem permanecer. Seria o mesmo que o conceito de subgrafo induzido.
\item \textbf{RLF}: Este algoritmo possui uma estratégia distinta: colore sequencialmente os conjuntos estáveis, conforme vai construindo-os.
\end{itemize}

% TODO comentar sobre o desempenho geral deles: O(n^3) RLF e O(n^2) os outros dois.

A heurística RLF é utilizada como referência tanto na questão da qualidade da heurística como no desempenho. O artigo, portanto, estuda uma implementação mais eficiente: Lazy RLF. O artigo, também, estuda uma estrutura de dados diferente para a representação do grafo.

\subsection{Recursive Largest First - RLF.}

Este algoritmo possui a complexidade no pior caso de $\mathcal{O}(n^3)$. Ele é dividido em 2 partes, sendo o seu coração a procura do conjunto estável (o conjunto no qual os vértices, dois a dois, não são adjacentes).

\subsubsection{Algoritmo 1.}

Consiste em chamar uma função para encontrar o conjunto estável, enquanto o subgrafo induzido possuir mais de 0 vértice (ou seja, enquanto houver um subgrafo...). Este algoritmo é responsável, a cada iteração, de atualizar o contador das cores utilizadas.

\subsubsection{Algoritmo 2.}

Na segunda parte é onde ocorre a computação relevante. Inicialmente podemos declarar:

\begin{itemize}
\item $G[X]$ como o subgrafo induzido pelo conjunto X;
\item $\delta (v)$ é o conjunto dos vértices adjacentes à v no $G[X \cup v]$;
\item $d_X(v)$ é o grau de $v$ induzido por $X$ sendo $d_X(v) = |\delta (v)|$;
\item $P$ é o conjunto de vértices que ainda não foram coloridos;
\item $U$ é o conjunto de vértice que não podem ser adicionados ao conjunto estável;
\end{itemize}

Tem-se inicialmente que $P = V$ e que $U = \emptyset$ e o grau induzido em $U$ para todos os vértices é 0. O procedimento seleciona sucessivamente o vértive $v$ para ser colorido, move os seus vizinhos $\delta_p (v)$ de $P$ para $U$, reduz o grafo de entrada G e atualiza o grau de cada vértice adjacente ao vértice movido de $P$ para $U$ induzido por $U$.

\subsubsection{Complexidade.}

A complexidade do algoritmo é de $\mathcal{O}(n^3)$, como mencionado anteriormente. A função ``findStableSet'' é executada k vezes. Entretanto, é interessante notar que cada vértice é selecionado uma única vez. O vértice é removido juntamente com suas respectivas arestas. Assim, cada aresta é removida uma única vez, o que implica em uma complexidade de $\mathcal{O}(m)$ nesta operação.

O loop aninhado da linha 10-12, no pior caso, pode ter a complexidade de $\mathcal{O}(n^2)$. Considerando o grau máximo do vértice em G, a complexidade pode ser de $\mathcal{O}(\Delta^2)$. Como o loop aninhado é executado uma única vez para cada vértice já que na linha 4 um vértice só pode ser escolhido uma única vez, a complexidade é de $\mathcal{O}(n\Delta^2)$. Todas as demais operações podem ser implementadas em tempo constante, exceto as linhas 2 e 4 que possuem a complixadade de $\mathcal{O}(n)$.

Em resumo, a complexidade total pode ser determinada em $\mathcal{O}(m + n\Delta^2)$.

\subsection{Lazy Recursive Largest First.}

O algoritmo RLF tem como principal característica a escolha do vértice com o grau máximo induzido por $U$. Para tanto, cada vez que um vértice é selecionado, uma array é atualizada e é visitado todos os seus vizinhos. Caso seja corretamente mantido o grau induzido de cada vértice a seleção do vértice é linear em $n$.

Uma alteração possível para o algoritmo, tendo isto posto, é utilizar uma estratégia diferente para selecionar o vértice. Ao invés de manter uma array de grau de vértices e modificá-la a cada seleção de vértices, pode-se computar $d_U(v)$ de forma ``preguiçosa'' utilizando a seguinte proposição:

Sabendo que $d_U^{max} = max\{d_U(v) | v \in P\}$ e que $\bar{d}_U$ é o maior grau induzido por $U$ encontrado, até o momento, na chamada à função findStableSet:
\begin{lazy}
  Um vértice $v$ com o $d_V(v) < \bar{d}_U$ tem $\bar{d_V} < d_U^{max}$.
\end{lazy}

\begin{proof}
  O grafo induzido pelo conjunto de vértices que ainda não foram coloridos temos que $\delta_v(v) = \delta_P \cup \delta_U(v)$ e portanto temos que $d_V(v) = d_P(v) + d_U(v)$. Assim, se $d_V(v) < \bar{d}_U$ então $d_U(v) \leq d_V(v) < \bar{d}_U \leq d_U^{max}$.
\end{proof}

Como consequência prática, temos que caso seja encontrado um vértice $\bar{d}_U$ é possível evitar a computação de  $d_U$ para todos os vértices com $d_V(v) < \bar{d}_U$. E, também, ao mudar a forma como $d_U(v)$ é computado (começando de $d_V(v)$ e decrescendo o seu valor ao invés de começar de 0 e incrementando-o), sempre que $d_U(v) < \bar{d}_U$ é computado quase que $d_V(v) - \bar{d}_U$ operações ao invés de $d_V(v)$.
Obviamente, quanto maior for $\bar{d}_U$ maior é a economia de computação. Além do mais, caso seja iniciado o cálculo com o vértice de máximo grau em G há grandes chaces de haver um valor alto para $\bar{d}_U(v)$.

% TODO Descrição dos algoritmos 3 e 4

Tendo em mente a proposição 1, é proposto 2 novos algoritmos: o algoritmo 3 é uma modificação do algoritmo 2. Ele remove as linhas 2 e 10-12 e altera a linha 4 (a forma como se obtém o vértice de grau máximo induzido por $U$), introduzindo a função ``lazySelectVertex'', baseada na proposição 1. Já o algoritmo 4 é a implementação da função ``lazySelectVertex''.

\subsubsection{Complexidade.}

% TODO detalhar

A complexidade, no pior caso, é de $\mathcal{O}(m + n^2\Delta)$.

\subsection{Estrutura de Dados.}

% TODO Descrever com mais clareza.

Usa lista de adjacência pois ela é eficiente: tem a remoção de aresta e vértices em tempo constante. a lista de adjacência possui com os vizinhos de $i$ possui 2 campos: um ponteiro para cada vizinho $j$ na lista de vértices e um ponteiro para $i$ na lista de adjacência de $j$. Para minimizar os problemas de ``perca de cache'' é utilizado uma implementação baseada em arrays da lista ligada. O problema de ``perca de cache'' é minimizado pois as arrays são alocadas de forma contígua na memória.


% \section{Comentários Jefferson.}

% \begin{itemize}
% \item \textbf{Qualidade}: Como eu garanto que um determinado algoritmo encontrou a resposta correta? Faço uma verificação manual? Se sim, há casos que eu não posso realizar essa verificação (por exemplo, quando o número de vértice é muito grande e humanamente impossível colorí-los).
%   \begin{itemize}
%   \item \textit{Após um pouco de pesquisa o que eu pude notar é que o algoritmo deve ser provado correto (formalmente) para que seja possível garantir que ele acha a solução correta.}
%   \end{itemize}
% \item aparentemente há uma classe de algoritmos \lq\lq{}construtivos\rq\rq{} para o GCP. Se existe tal classe, deve haver outras classes, não?
% \end{itemize}

\nocite{*}
\bibliographystyle{ieeetr}
\bibliography{efficiency_issues_in_the_RLF_heuristic_for_graph_coloring}

\end{document}