%% Copyright (C)  2013 Jefferson Campos.
%% Permission is granted to copy, distribute and/or modify this document
%% under the terms of the GNU Free Documentation License, Version 1.3
%% or any later version published by the Free Software Foundation;
%% with no Invariant Sections, no Front-Cover Texts, and no Back-Cover Texts.
%% A copy of the license is included in the section entitled "GNU
%% Free Documentation License".

\documentclass[a4paper,12pt]{article}
\usepackage[brazilian]{babel}
\usepackage[utf8]{inputenc}
\usepackage[T1]{fontenc}
\usepackage{hyperref}
\usepackage{url}
\usepackage{amssymb}
\usepackage[pdftex]{graphicx}
\newcommand{\HRule}{\rule{\linewidth}{0.5mm}}

\begin{document}

% Nesta seção serão capturadas as principais idéias do texto
\section{Resumo.}

GCP = Graph Coloring Problem

O problema de coloração de vértices pode ter, ao menos, duas versões: otimização e decisão.

Existem basicamente 3 heurísitcas utilizadas para o GCP, que apesar de terem um desempenho razoável não garantem uma aproximação da solução/qualidade da solução:

\begin{itemize}
\item \textbf{simplest}: ordena os vértices do grafo pelo seu grau e sequencialmente colore cada vértice, de acordo com a ordem pré-estabelecida, utilizando a menor cor possível.
\item \textbf{DSATUR}: Possui o mesmo \lq\lq{}espírito\rq\rq{} do algoritmo \textbf{\lq\lq{}simplest\rq\rq{}} porém com o diferencial na forma de ordenação dos vértices. Esta ordenação utiliza-se do \lq\lq{}grau de saturação\rq\rq{}, isto é, que tem o maior grau induzido pelos vértices já coloridos. % Minha percepção: o fato de induzir pelos vértices já coloridos é justamente retirar os vértices já coloridos porém as demais arestas devem permanecer. Seria o mesmo que o conceito de subgrafo induzido.
\item \textbf{RLF}: Este algoritmo possui uma estratégia distinta: colore sequencialmente os conjuntos estáveis, conforme vai construindo-os.
\end{itemize}

% TODO comentar sobre o desempenho geral deles: O(n^3) RLF e O(n^2) os outros dois.

A heurística RLF é utilizada como referência tanto na questão da qualidade da heurística como no desempenho. O artigo, portanto, estuda uma implementação mais eficiente: Lazy RLF. O artigo, também, estuda uma estrutura de dados diferente para a representação do grafo.

\subsection{Recursive Largest First - RLF.}

Este algoritmo possui a complexidade no pior caso de $\mathcal{O}(n^3)$. Ele é dividido em 2 partes, sendo o seu coração a procura do conjunto estável (o conjunto no qual os vértices, dois a dois, não são adjacentes).

\subsubsection{Algoritmo 1.}

Consiste em chamar uma função para encontrar o conjunto estável, enquanto o subgrafo induzido possuir mais de 0 vértice (ou seja, enquanto houver um subgrafo...). Este algoritmo é responsável, a cada iteração, de atualizar o contador das cores utilizadas.

\subsubsection{Algoritmo 2.}

Na segunda parte é onde ocorre a computação relevante. Inicialmente podemos declarar:

\begin{itemize}
\item $G[X]$ como o subgrafo induzido pelo conjunto X;
\item $\delta (v)$ é o conjunto dos vértices adjancetes à v;
\item $d_X(v)$ é o grau de $v$ induzido por $X$ sendo $d_X(v) = | \delta (v)|$;
\item $P$ é o conjunto de vértices que ainda não foram coloridos;
\item $U$ é o conjunto de vértice que não podem ser adicionados ao conjunto estável;
\end{itemize}

Tem-se inicialmente que $P = V$ e que $U = \emptyset$.

\subsubsection{Complexidade.}

\section{Comentários Jefferson.}

\begin{itemize}
\item Como eu garanto que um determinado algoritmo encontrou a resposta correta? Faço uma verificação manual? Se sim, há casos que eu não posso realizar essa verificação (por exemplo, quando o número de vértice é muito grande e humanamente impossível colorí-los).
\item aparentemente há uma classe de algoritmos \lq\lq{}construtivos\rq\rq{} para o GCP. Se existe tal classe, deve haver outras classes, não?
\end{itemize}

\nocite{*}
\bibliographystyle{ieeetr}
\bibliography{efficiency_issues_in_the_RLF_heuristic_for_graph_coloring}

\end{document}