\documentclass[12pt]{article}
\usepackage[brazilian]{babel}
\usepackage[utf8]{inputenc}
%\usepackage{mcmacros}
\usepackage{fullpage}
\usepackage{amsmath}
\usepackage{amssymb}
\usepackage{graphicx}
\usepackage{subfigure}
\usepackage{eepic}
\usepackage{setspace}
\usepackage{url}
\usepackage{pgfgantt, tikz}
\usepackage[pdftex]{hyperref}

\newcommand{\mysection}[1]{\section*{#1}} 
\newcommand{\mysubsection}[1]{\subsection*{#1}} 
\newcommand{\cqd}{\begin{flushright} \small $\square$ \end{flushright}}
\newtheorem{thm}{Teorema}
\newtheorem{cor}{Corolário}
\newtheorem{lem}{Lema}
\newtheorem{prp}{Proposição}
\newtheorem{cnj}{Conjetura}

\newcommand{\theolike}[2]{\textbf{#1} \textit{#2}}
\newcommand{\f}{$\blacksquare$}
\newcommand{\BigO}[1]{\ensuremath{\operatorname{O}\left(#1\right)}}

\author{\textbf{Candidato:} Jefferson Luiz Oliveira de Campos\\
        \textbf{Orientadora:} Profa. Dra. Cândida Nunes da Silva}
\title{\textbf{Coloração de Vertices}}
\date{\today}

\doublespacing

\begin{document}

% Insere título do texto em página não numerada. 
\maketitle
\thispagestyle{empty}

% Numera páginas preliminares com numerais romanos.
\newpage
\pagenumbering{roman}
\tableofcontents

% Numera as páginas do texto com numerais arábicos normalmente.
\newpage
\pagenumbering{arabic}

%O projeto de pesquisa deve ser apresentado de maneira clara e resumida, ocupando, no máximo, 20 páginas datilografadas em espaço duplo. Deve compreender: resumo com no máximo 20 linhas, introdução e justificativa, com síntese da bibliografia fundamental; objetivos; plano de trabalho e cronograma de sua execução; material e métodos; forma de análise dos resultados. A responsabilidade pelo projeto é  do orientador, mas o aluno deve estar apto a discutir.

\section{Resumo}

Resumo...

\section{Introdução}

% Definições básicas de grafos
% * Grafo
% * vertices
% * arrestas
% * grafos simples
% * grafos completos

\subsection{Definição Grafos Básicas.}
Um grafo é um par ordenado que possui um conjunto de vertices, representados por $|V|$ e um conjunto de arrestas associadas à esses vértices representado por $|E|$.\\

Um grafo pode possuir laços, quando uma arresta possui início e fim no mesmo vértice assim como pode haver duas ou mais arrestas que possuem as pontas ligadas exatamente no mesmo vértice. Quando isso ocorre, chamamos esses vértices de vértices paralelos.\\

Um grafo simples é quando o grafo não possui laços e nem arrestas paralelas.\\ Já um grafo completo é quando todo para de vértices possui uma arresta ligando-os.\\

% Roteiro Cândida
\subsection{Definir o problema}

\subsection{Histórico}

% Historico do Hamilton na coloração.

O problema de mapas planares foi elucidado em 1852 por Willian Rowan Hamilton em uma carta para Augustus De Morgan.\\
Nesta carta, Hamilton conjecturava que seria possível colorir um mapa utilizando apenas 4 cores e não mais que isso. De Morgan não conseguiu apresentar uma prova para a conjectura, assim como Hamilton falhou nessa empreitada. Posteriormente foram aprensetadfas algumas provas à esse teorema mas foram descobertos diversos problemas com essas provas. Somente no século XX foi apresentada uma prova válida ao teorema.\\

%transcrever o trecho aqui...

\subsection{Relações entre Problemas.}

% Mapeamento entre o problema de coloração vertice/arresta/faces
% citar quais outros avanços - teorema das 5 cores.
É interessante notar que esse problema gerou interesse de diversos cientistas e que nesta busca pela prova foram gerados resultados interessantes que ajudaram no progresso de outras áreas dentro da Teoria dos Grafos. Um aspecto interessante deste problema é que ele pode ser mapeado para outros problemas e por consequência, os teoremas utilizados aqui podem ser utilizados nesses problemas. Podemos citar o problema de coloração vértices, o problema de coloração de arrestas e coloração de faces.

\subsection{Número de clique máxima conjunto estável/independente}

\subsubsection{Cardinalidade.}

A cardinalidade de um grafo é o número de vértices existentes no grafo.

\subsubsection{Estabilidade.}

Um conjunto é de vértices de um grafo é dito estável se nenhum para de vértices são adjacentes. Também é dito que se o conjunto estável não pode ser aumentado ele é maximal. Caso o conjunto estável seja o maior existente no grafo ele é chamado de conjunto estável máximo. O número estável, denotado por $\alpha$ é a cardinalidade de um conjunto estável.

\subsubsection{Clique.}
% mesma idéia do maximal em estabilidade.
Um clique é definido como um subgrafo completo do grafo original. O clique é máximal quando ele não pode ser aumentado. O clique é dito máximo se ao adicionar um novo vértice ao clique, ele deixa de ser um clique.

\subsubsection{Greedy Algorithm to Find Maximal Clique.}

Partindo de um clique qualquer, basta pegar um novo vértice ainda não testado do grafo e verificar se ele está conectado à todos os vértices do clique. Descarta-se o vértice se ele não estiver conectado à todos os vértices do clique. Caso contrário, adiciona-o ao clique. Este processo continua até que todos os vértices tenham sido testados.\\

% Preciso achar a fonte desta informação.
A complexidade deste algoritmo é de \BigO{n}.

\subsubsection{Software.}

Implementações disponíveis para o problema do máximo clique de um grafo.

\begin{enumerate}
\item \href{http://networkx.github.io/documentation/development/reference/generated/networkx.algorithms.approximation.clique.max_clique.html}{NetworkX}
\item \href{http://www.dcs.gla.ac.uk/~pat/maxClique/distribution/readme.txt}{maxClique}
\item \href{http://openopt.org/MCP}{OpenOpt}
\end{enumerate}


\subsection{NP-completo (Garey \& Johnson)}

\subsection{redução à sub-classe}

\subsection{algoritmos exatos}

\subsection{heurísticas disponíveis}


\section{Objetivos}
\label{sec:objetivos}

Objetivos...

\section{Plano de Trabalho e Cronograma}
\label{sec:plano}

\begin{ganttchart}[vgrid, hgrid, x unit=0.75cm, bar/.style={fill=black}]{12}
  \gantttitle{2013}{8}
  \gantttitle{2014}{4} \\

  \gantttitle{Abr}{1}
  \gantttitle{Mai}{1}
  \gantttitle{Jun}{1}
  \gantttitle{Jul}{1}
  \gantttitle{Ago}{1}
  \gantttitle{Set}{1}
  \gantttitle{Out}{1}
  \gantttitle{Nov}{1}
  \gantttitle{Dez}{1}
  \gantttitle{Jan}{1}
  \gantttitle{Fev}{1}
  \gantttitle{Mar}{1} \\

  \ganttbar{Revisão bibliográfica}{1}{12} \\
  \ganttbar{Estudo de \textsl{snarks}}{1}{2} \\
  \ganttbar{Estudo de grafos fluxo-críticos}{1}{2} \\
  \ganttbar{Investigação da Questão 1}{2}{6} \\
  \ganttbar{Escrita do relatório parcial}{6}{6} \\
  \ganttbar{Investigação da Questão 2}{7}{11} \\
  \ganttbar{Escrita do relatório final}{11}{12}
\end{ganttchart}

\section{Metodologia}

Por  ser um trabalho  de viés teórico, com  o objetivo  de investigar
propriedades  estruturais de  certos grafos,  a  metodologia utilizada
será a usual de propor  conjeturas baseadas na intuição obtida durante
a  investigação  de  exemplos  e tentar  posteriormente  prová-las  ou
refutá-las. Vislumbramos  que o uso de programas  de computadores para
testar a validade de  algumas propriedades para certos \textsl{snarks}
pode ser interessante no contexto  deste projeto como meio de auxiliar
nossa  intuição na proposição  de eventuais  conjeturas ou  mesmo para
resolvê-las.

\bibliographystyle{abbrv} 
\bibliography{frsi}
	
\section{Assinaturas}
\today

\vspace{1.5cm}
\begin{tabular}{ccc}
 &  & \\
 Jefferson Luiz Oliveira de Campos & \hspace{4cm} & Cândida Nunes da Silva \\
 (Candidato) & & (Orientadora)  \\
\end{tabular}

% FIXME bibtex not working!!!
% \bibliography{jefferson_qualif}
% \nocite{MUNARI:2006}

% Bibliografia Utilizada até o momento:
% * Boundy & Murty - Livro Base
% * Paulo Oswaldo Boaventura Netto - Grafos
% * Ruy Eduardo Campelo & Nelson Maculan - algoritmos e heurísticas

\end{document}