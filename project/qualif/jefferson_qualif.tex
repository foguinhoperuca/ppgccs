\documentclass[12pt]{article}
\usepackage[brazilian]{babel}
\usepackage[utf8]{inputenc}
%\usepackage{mcmacros}
\usepackage{fullpage}
\usepackage{amsmath}
\usepackage{amssymb}
\usepackage{graphicx}
\usepackage{subfigure}
\usepackage{eepic}
\usepackage{setspace}
\usepackage{url}
\usepackage{pgfgantt, tikz}

\newcommand{\mysection}[1]{\section*{#1}} 
\newcommand{\mysubsection}[1]{\subsection*{#1}} 
\newcommand{\cqd}{\begin{flushright} \small $\square$ \end{flushright}}
\newtheorem{thm}{Teorema}
\newtheorem{cor}{Corolário}
\newtheorem{lem}{Lema}
\newtheorem{prp}{Proposição}
\newtheorem{cnj}{Conjetura}

\newcommand{\theolike}[2]{\textbf{#1} \textit{#2}}

\newcommand{\f}{$\blacksquare$}    

\author{\textbf{Candidato:} Jefferson Luiz Oliveira de Campos\\
        \textbf{Orientadora:} Profa. Dra. Cândida Nunes da Silva}
\title{\textbf{Coloração de Vertices}}
\date{\today}

\doublespacing

\begin{document}

% Insere título do texto em página não numerada. 
\maketitle
\thispagestyle{empty}

% Numera páginas preliminares com numerais romanos.
\newpage
\pagenumbering{roman}
\tableofcontents

% Numera as páginas do texto com numerais arábicos normalmente.
\newpage
\pagenumbering{arabic}

%O projeto de pesquisa deve ser apresentado de maneira clara e resumida, ocupando, no máximo, 20 páginas datilografadas em espaço duplo. Deve compreender: resumo com no máximo 20 linhas, introdução e justificativa, com síntese da bibliografia fundamental; objetivos; plano de trabalho e cronograma de sua execução; material e métodos; forma de análise dos resultados. A responsabilidade pelo projeto é  do orientador, mas o aluno deve estar apto a discutir.

\section{Resumo}

Resumo...

\section{Introdução}

Intro...


\section{Objetivos}
\label{sec:objetivos}

Objetivos... \cite{teoria_sistema_CAPRA}

\section{Plano de Trabalho e Cronograma}
\label{sec:plano}

\begin{ganttchart}[vgrid, hgrid, x unit=0.75cm, bar/.style={fill=black}]{12}
  \gantttitle{2013}{8}
  \gantttitle{2014}{4} \\

  \gantttitle{Abr}{1}
  \gantttitle{Mai}{1}
  \gantttitle{Jun}{1}
  \gantttitle{Jul}{1}
  \gantttitle{Ago}{1}
  \gantttitle{Set}{1}
  \gantttitle{Out}{1}
  \gantttitle{Nov}{1}
  \gantttitle{Dez}{1}
  \gantttitle{Jan}{1}
  \gantttitle{Fev}{1}
  \gantttitle{Mar}{1} \\

  \ganttbar{Revisão bibliográfica}{1}{12} \\
  \ganttbar{Estudo de \textsl{snarks}}{1}{2} \\
  \ganttbar{Estudo de grafos fluxo-críticos}{1}{2} \\
  \ganttbar{Investigação da Questão 1}{2}{6} \\
  \ganttbar{Escrita do relatório parcial}{6}{6} \\
  \ganttbar{Investigação da Questão 2}{7}{11} \\
  \ganttbar{Escrita do relatório final}{11}{12}
\end{ganttchart}

\section{Metodologia}

Por  ser um trabalho  de viés teórico, com  o objetivo  de investigar
propriedades  estruturais de  certos grafos,  a  metodologia utilizada
será a usual de propor  conjeturas baseadas na intuição obtida durante
a  investigação  de  exemplos  e tentar  posteriormente  prová-las  ou
refutá-las. Vislumbramos  que o uso de programas  de computadores para
testar a validade de  algumas propriedades para certos \textsl{snarks}
pode ser interessante no contexto  deste projeto como meio de auxiliar
nossa  intuição na proposição  de eventuais  conjeturas ou  mesmo para
resolvê-las.

\bibliographystyle{abbrv} 
\bibliography{frsi}
	
\section{Assinaturas}
\today

\vspace{1.5cm}
\begin{tabular}{ccc}
 &  & \\
 Jefferson Luiz Oliveira de Campos & \hspace{4cm} & Cândida Nunes da Silva \\
 (Candidato) & & (Orientadora)  \\
\end{tabular}

\bibliography{jefferson_qualif}


\end{document}